\documentclass[a4paper]{article}

\renewcommand{\thesubsubsection}{\arabic{subsubsection}}

\title{IN2349 ADLR: Project Ideas}
\begin{document}
\maketitle

Here you can find a number of ideas for projects we collected. Take this as an inspiration for your own project. Some of the ideas are rather "big", meaning they could result in more than one project. After you have registered your team (including a draft proposal) you will discuss the extent of your final proposal with your assigned tutor.

\subsubsection{Closing the Sim-to-Real Loop: Adapting Simulation Randomization with Real World Experience}
In~\cite{Chebotar2018} an efficient method for solving the sim2real problem by iteratively adapting the simulation parameters to the real system is proposed.
\begin{itemize}
\item Change the experimental setup to a sim2sim setting because no real robot is available (i.e., try to adapt one simulation to a given one with unknown parameters).
\item Investigate the convergence of estimated parameters.
\item Investigate the influence of a broader starting distribution on the final performance.
\item Investigate how re-using learned policies from previous iterations for initialization of the new network shortens training time and restrains the final performance.
\end{itemize}

\subsubsection{Comparing Robustness of Learned (Quadruped Walking) Policies from Different Learning Algorithms}
\cite{Haarnoja2018b} claim that RL algorithms with stochastic policies such as SAC are able to express a much more diverse behavior compared to deterministic algorithms like DDPG. As a consequence, stochastic policies should be far more robust when deployed in a perturbed environment.
\begin{itemize}
\item Investigate how stochastic policies (SAC, PPO) perform in novel environments when compared to deterministic algorithms like DDPG.
\item Investigate the influence of different distributions used for SAC on the performance in novel environments. Early version of SAC employ GMMs which are discarded in favor of a single Gaussian in later papers.
\item Investigate the methods for a quadruped with more complex powertrain dynamics than the one used by~\cite{Haarnoja2018b}.
\end{itemize}

\subsubsection{Investigate Reinforcement Learning with Augmented Networks}
In recent years, different methods to tackle problems in robotics have emerged. Especially for POMDPs or complex tasks that employ domain randomization for the sim2real transfer, the incorporation of LSTM cells into the network architecture leads to impressive results as shown by~\cite{OpenAI2019}. One downside to using recurrent architecture is the increased training complexity. Other methods like observation stacking or explicit physical parameter estimation using supervised learning are far easier to implement and train.
\begin{itemize}
\item ~\cite{OpenAI2019} claim that memory augmented models exhibit 'meta-learning' behavior at test time, meaning that the policy trials the environment to adapt parameters before starting to work on the desired task itself. Investigate these claims in an environment inspired by a physical system.
\item Investigate how LSTM cells, observation stacking and explicit parameter estimation perform in the same partially observable physics simulation when simple domain randomization is employed. Some task that require precise knowledge of the domain parameterization are particularly interesting
\end{itemize}

\subsubsection{Generalized Hindsight for Reinforcement Learning}
Generalized Hindsight for Reinforcement Learning presented by~\cite{Li2020} combines Multi-Task Deep Reinforcement Learning with Sample-Efficient Learning.
\begin{itemize}
\item Design a physically inspired and meaningful task that can be used for Multi-Task learning, e.g. a more complex version of the ant environment used by~\cite{Li2020}. This simulation can be used to learn different gaits, pronking as well as yawrate control at different speeds.
\end{itemize}

\subsubsection{Comparing different methods for uncertainty estimation}
\label{sec:uncertainty-estimation}
Interesting methods include (but are not limited to) MC-Dropoout \cite{Gal2016} and Normalizing Flows \cite{louizos2017multiplicative}. Thsese methods could be compared in vastly different settings.
\begin{itemize}
    \item Investigate how the uncertainty estimation changes during the training process (relevant to RL since we generally don’t update the networks until convergence before collecting more data).
    \item Investigate which methods are best suited for active learning in the framework proposed by \cite{gal2017active}.
    \item Investigate which methods perform best for DQNs in simple environments similar to the work by \cite{touati2018randomized}.
    \item Come up with your own ideas.
\end{itemize}

\subsubsection{Curiosity-Driven Learning}
A good starting point for projects in this area is the paper by \textit{Large-Scale Study of Curiosity-Driven Learning} \cite{burda2018largescale}. Individual projects could first reporduce some of the experiments and then:
\begin{itemize}
    \item Compare with other curiosity "measures" (count based, predict next state, predict past action, random network distillation, reachability, goal based, and/or your own ideas).
    \item Extend the authors comparison to other sets of environments. One key question is whether or not there are certain characteristics that a task needs to fulfill in order to profit from curiosity-driven learning.
\end{itemize}

\subsubsection{Geometric Representations in Reinforcement Learning}
Note: Requires previous experience with GNNs \cite{kipf2016semisupervised}.
\begin{itemize}
    \item Modify PyBullet environments (Hopper, Walker, HalfCheetah, Ant) such that the observations contain a graph representing the robot. Use message passing network(s) in addition or instead of the MLP for value/Q function and policy in standard algorithms like PPO \cite{Schulman2017} or SAC \cite{Haarnoja2018a}.
\end{itemize}

\subsubsection{AlphaZero Implementation \& Tweaks}
Deep dive into the AlphaGo/AlphaZero algorithm~\cite{Silver2016, Silver2017, Silver2017a}. Pick a very small game for feasible computational \& time cost and quick experimentation. Either implement from scratch or maybe get very familiar with an existing implementation. Then try variations or possible algorithm improvements.

Modification ideas:
\begin{itemize}
  \item Refined handling of explore/exploit tradeoff in MCTS by adding Uncertainty Estimation (see section~\ref{sec:uncertainty-estimation}) in value \& prior networks for better upper confidence bounds in PUCT. Can this speed up discovery of good strategies?
  \item Comparison to model-free RL: Does AlphaZero learn better/worse? Does this depend on environment (game) characteristics? What metrics should be compared here?
\end{itemize}

\bibliographystyle{apalike}
\bibliography{minimal-research}

\end{document}
