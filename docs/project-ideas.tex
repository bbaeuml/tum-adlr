\documentclass[a4paper]{article}

\renewcommand{\thesubsubsection}{\arabic{subsubsection}}

\title{IN2349 ADLR: Project Ideas}
\begin{document}
\maketitle

Here you can find a number of ideas for projects we collected. Take this as an inspiration for your own project. Some of the ideas are rather "big", meaning they could result in more than one project. After you have registered your team (including a draft proposal) you will discuss the extent of your final proposal with your assigned tutor.

\subsubsection{Closing the Sim-to-Real Loop: Adapting Simulation Randomization with Real World Experience}
\begin{itemize}
\item In~\cite{Chebotar2018} an efficient method for solving the sim2real by iteratively adapting the simulation parameters to the real system is proposed.
\item Change the experimental setup to a sim2sim setting because no real robot is available (i.e., try to adapt one simulation to a given one with unknown parameters).
\item Investigate the convergence of estimated parameters.
\item Investigate the influence of a broader starting distribution on the final performance.
\item Investigate the influence of fully training the policy in every iteration vs. iterative training (the paper only covers full retraining).
\end{itemize}

\subsubsection{Comparing robustness of learned (quadruped walking) policies from different learning algorithms}
\begin{itemize}
\item Investigate how stochastic policies (SAC, PPO) perform in novel environments when compared to deterministic algorithms like DDPG. SAC and PPO claim to be far more robust.
\item Investigate the influence of different distributions used for SAC (first SAC paper used GMMs) on the performance in novel environments.
\item Investigate the methods for a quadruped with a more complex powertrain dynamics than the one used in the SAC paper. This can be combined with the Combinable with 3, 4b
\end{itemize}

\subsubsection{Comparing the different methods for uncertainty estimation}
Interesting methods include (but are limited to) MC-Dropoout \cite{Gal2016} and Normalizing Flows \cite{louizos2017multiplicative}. Thsese methods could be compared in vastly different settings.
\begin{itemize}
    \item Investigate how the uncertainty estimation changes during the training process (relevant to RL since we generally don’t update the networks until convergence before collecting more data).
    \item Investigate which methods are best suited for active learning in the framework proposed by \cite{gal2017active}.
    \item Investigate which methods perform best for DQNs in simple environments similar to the work by \cite{touati2018randomized}.
    \item Come up with your own ideas.
\end{itemize}

\subsubsection{Curiosity-Driven Learning}
A good starting point for projects in this area is the paper by \textit{Large-Scale Study of Curiosity-Driven Learning} \cite{burda2018largescale}. Individual projects could first reporduce some of the experiments and then:
\begin{itemize}
    \item Compare with other curiosity "measures" (count based, predict next state, predict past action, random network distillation, reachability, goal based, and/or your own ideas).
    \item Extend the authors comparison to other sets of environments. One key question is whether or not there are certain characteristics that a task needs to fulfill in order to profit from curiosity-driven learning.
\end{itemize}

\subsubsection{Geometric Representations in Reinforcement Learning}
Note: Requires previous experience with GNNs \cite{kipf2016semisupervised}.
\begin{itemize}
    \item Modify PyBullet environments (Hopper, Walker, HalfCheetah, Ant) such that the observations contain a graph representing the robot. Use message passing network(s) in addition or instead of the MLP for value/Q function and policy in standard algorithms like PPO \cite{Schulman2017} or SAC \cite{Haarnoja2018a}.
\end{itemize}


\bibliographystyle{apalike}
\bibliography{minimal-research}

\end{document}
