\documentclass[a4paper]{article}
\usepackage{hyperref}

\renewcommand{\thesubsubsection}{\arabic{subsubsection}}
\title{IN2349 ADLR: Project Ideas}
\author{}

\begin{document}
\maketitle

Here you can find a number of ideas for projects we collected. 
Take this as an inspiration for your own project. 
Some of the ideas are rather "big", meaning they could result in more than one project. 
After you have registered your team (including a draft proposal) you will discuss the extent of your final proposal with your assigned tutor.


% \subsubsection{Comparing Methods for Uncertainty Estimation}
% Interesting methods include MC-Dropoout \cite{Gal2016}, Bootstrapping \cite{Osband2018}, and Normalizing Flows \cite{louizos2017multiplicative}. These methods could be compared in vastly different settings.
% \begin{itemize}
%   \item Investigate how the uncertainty estimation changes during the training process (relevant to RL since we generally don’t update the networks until convergence before collecting more data).
%   \item Investigate which methods are best suited for active learning in the framework proposed by \cite{gal2017active}.
%   \item Investigate which methods perform best for DQNs in simple environments similar (\cite{BSuit2020}, \cite{touati2018randomized}).
%   \item Come up with your own ideas.
% \end{itemize}

\subsubsection{Offline Datasets for Reinforcement Learning}
Offline (Batch) RL has recently gained more attention, e.g. \cite{Agarwal2019}, \cite{nair2020}, \cite{AWOpt2021} (cf. \url{github.com/rail-berkeley/d4rl}, \url{github.com/deepmind/deepmind-research/tree/master/rl_unplugged}).
\begin{itemize}
  \item Compare different Batch RL algorithms.
  \item Test new environments.
  \item Benchmark against online algorithms.
\end{itemize}


\subsubsection{Learning to Fly}
Fixed-wing VTOL (Vertical Take-Off and Landing) drones are highly efficient in long-range flight, but difficult to control during gusty landing phases. Xu et al., \cite{LearningToFly} presented an error convolution input enabling the learned controller to adapt for different airframes.
\begin{itemize}
  \item Test the approach on VTOL drones with less propeller actuation but control flaps. 
  \item Expand the idea to continuous action spaces.
  \item Investigate what sensor readings could be added to the state space to increase stability in gusty conditions.
  \item Utilize our drone model \cite{BionicVTOL} implemented in Julia as the high-efficient, flexible and dynamic programming language of the future. 
\end{itemize}


\subsubsection{Decision Transformer}
The Transformer architecture is extremely effective for language models, and has shown to promising results on computer vision tasks.
Therefore, researches are exploring the application of those models for Reinforcement Learning (in particular, on offline datasets) \cite{DecisionTransformer2021}, \cite{OneBigSequence2021}. In your project you could, for example:
\begin{itemize}
  \item Train a Decision Transformer in the cloud.
  \item Analyze evaluation performance in detail, improve through different planning approaches.
  \item Adjust intermediate training signals to improve performance.
\end{itemize}

% \subsubsection{Geometric Representations in Reinforcement Learning}
% Note: Requires previous experience with GNNs \cite{kipf2016semisupervised}.
% \begin{itemize}
%   \item Similar to \cite{Wang2018nervenet}. Modify PyBullet environments (Hopper, Walker, HalfCheetah, Ant) such that the observations contain a graph representing the robot.
%   \item Use message passing network(s) in addition or instead of the MLP for value/Q function and policy in standard algorithms like PPO \cite{Schulman2017} or SAC \cite{Haarnoja2018a}.
% \end{itemize}

%\subsubsection{Closing the Sim-to-Real Loop: Adapting Simulation Randomization with Real World Experience}
%In~\cite{Chebotar2018} an efficient method for solving the sim2real problem by iteratively adapting the simulation parameters to the real system is proposed.
%\begin{itemize}
%  \item Change the experimental setup to a sim2sim setting because no real robot is available (i.e., try to adapt one simulation to a given one with unknown parameters).
%  \item Investigate the convergence of estimated parameters.
%  \item Investigate the influence of a broader starting distribution on the final performance.
%  \item Investigate how re-using learned policies from previous iterations for initialization of the new network shortens training time and restrains the final performance.
%\end{itemize}

%\subsubsection{Meta Reinforcement Learning for Sim-to-real Domain Adaptation}
%\cite{Arndt19} show that the sim2real problem can be tackled by applying Meta RL to learned low dimensional projections of the action space.
%\begin{itemize}
%  \item Implement the algorithm
%  \item Change the experimental setup to a sim2sim setting because no real robot is available (i.e., try to adapt one simulation to a given one with unknown parameters).
%  \item Investigate how this framework can be applied or extended to classic closed-loop robot control
%\end{itemize}

% \subsubsection{Solving Complex Sparse Reinforcement Learning Tasks}
% When defining Reinforcement Learning Tasks for robots, it is often desirable to stick to sparse rewards in order to avoid reward shaping. Not only does is it ease the setup since a task can be solely defined by its final desired outcome, but the optimal policy can be found "outside of the box" without a human prior given through the reward function. Unfortunately, in big state spaces random exploration is not able to find these sparse success signals. Therefore, \cite{riedmiller2018} introduce SAC-X.
% \begin{itemize}
%   \item Implement the algorithm
%   \item Investigate how the presented method can be used for finer and more dexterous object manipulation e.g. with a hand.
% \end{itemize}

\subsubsection{Learning the Inverse Kinematics}
Look at the possibilities for representing inverse problems with neural networks.
\textit{Analyzing Inverse Problems with Invertible Neural Networks}~\cite{Ardizzone2018}
compare different flavors of GANs, VAEs and INN(theirs) for inverse problems in general.
Extend their simple robotic example of a planner arm to 3D / more DoFs / multiple TCPs.
Unlike in computer vision, for the robot kinematic we have solid metrics to describe how well the generation task was performed.
How can we use this knowledge to our advantage?

\begin{itemize}
  \item What is the best approach to represent the high dimensional nullspaces for complex robot geometries?
  \item ~\cite{Lembono2021} use an ensemble of GANs to reduce the impact of mode collapse.
        What other options do we have to improve the generative model?
  \item Predict not only the position of the TCP but also its rotation. 
  How can one best represent the $SO(3)$ ~\cite{Zhou2018}.
  \item How to measure the performance if the real nullspace is not known?
\end{itemize}

\subsubsection{Harnessing Reinforcement Learning for Neural Motion Planning}
Motion planning for a planar robotic arm from a start configuration to a cartesian goal position.
Comparison between DDPG, DDPG+HER, and DDPG-MP(theirs) ~\cite{Jurgenson2019}.
They use RRT* to generate expert knowledge in difficult cases, where random exploration does not find a feasible solution.
\begin{itemize}
  \item Modify the code and try it for different robots and environments.
  \item Is this expert knowledge necessary or can this also be achieved with a well designed curriculum?
  \item look at modern approaches to represent the environment in which the robot moves (ie. Basis Points Set ~\cite{Prokudin2019}, Example of PointNet for Motion Planning ~\cite{Strudel2020})
  \item They state that supervised learning is inferior to RL for this problem because of the insufficient data on the boundary of the obstacles. Is it possible to achieve similar results by tweaking the distribution of the supervised examples?
\end{itemize}


\subsubsection{Learning to Optimize Motion Planning}
Explore the ideas proposed by "Learning to Optimize" ~\cite{LiM16b} in the context of optimization based motion planning \cite{Zucker2013}.
Can Reinforcement Learning guide an optimizer to speed up robotic path planning?
How does it relate to the approach of "Unsupervised Path Regression Networks" ~\cite{Pandy2020}.
Can we combine those ideas?

\begin{itemize}
  \item Set up an optimizer for a simple robot (with help from the tutor)
  \item Test ideas to guide the optimization problem of motion planning
  \item What are advantages of this hybrid approach over using RL directly on motion planning?
\end{itemize}

\subsubsection{Trajectory planning with moving obstacles}
Drones not only have to plan flight paths through static environments, but also avoid collisions with dynamic objects.
\begin{itemize}
  \item Set up a simple 2D environment in which an agent can navigate through moving obstacles.
  \item Come up with a state representation that reflects the motion of the obstacles and allows for a changing number of objects.
  \item Apply neural motion planning to the problem and provide a possible trajectory in a time-efficient process.
  \item Optional: Extend the method to fixed-wing drones in three-dimensional space and test your approach in our own simulator \cite{BionicVTOL} or in the DodgeDrone challenge \cite{DodgeDroneChallenge}.
  \item Optional: Experiment with additional uncertainties in the motion prediction of the collision objects.
\end{itemize}


\subsubsection{Recurrent Off-Policy Reinforcement Learning in POMDPs}
In partially observable Markov decision processes (POMDPs), a RL agent has to be equipped with some sort of memory in order to be able to act optimally. A well known method addressing this issue is to encode the history of observations by Recurrent Neural Networks (RNNs).
For the class of Off-Policy methods, \cite{heess2015memory} combine RNNs with the DDPG algorithm and \cite{kapturowski2018recurrent} study the interplay of DQN-based algorithms with recurrent experience replay.
Based on these works:
\begin{itemize}
  \item Choose POMDP environments that require the use of memory to be solved optimally.
  \item Implement a recurrent version of the SAC algorithm (\cite{Haarnoja2018a}).
  \item Assess the effect of different design choices and hyperparameters (e.g. hidden state initialization strategy in the experience replay buffer, truncated BPTT, ...)
\end{itemize}

%\subsubsection{Differentiable Bayesian Filters}
%\textit{*Prior knowledge of bayesian filters highly recommended*}\\
%Differentiable filters are a promising approach for combining the algorithmic structure of bayesian filter techniques with the power of learning-based methods (for an overview over existing methods, see e.g. \cite{kloss2021train}). Importantly, differentiable filters offer a systematic way of dealing with aleatoric uncertainty in state estimation problems.
%\begin{itemize}
%  \item Implement a \textit{differentiable} filter of your choice, for example EKF, UKF or Particle Filter.
%  \item Consider the simple tracking experiment VI from \cite{kloss2021train}. Implement interesting modification to the experiment. For example:
%        \begin{itemize}
%          \item use the distance to beacons instead of images as measurements
%          \item implement collisions
%          \item additionally estimate parameters from the system dynamics on-the-fly
%        \end{itemize}
%  \item Compare your filter to recurrent neural networks and discuss the benefits.
%\end{itemize}

\subsubsection{Exploring Munchausen Reinforcement Learning}
Recently, \cite{vieillard2020munchausen} proposed an appealingly simple, yet surprisingly effective extension to DQN; using the policy for bootstrapping. Exploring the implications of this idea, projects could for example address a (sub) set of the following problems:

\begin{itemize}
  \item Extend the idea to continuous action spaces (e.g. by augmenting SAC).
  \item Apply Munchausen RL to robotic tasks and see what it brings to the table. Does it improve the baselines? If yes, under what circumstances? If no, investigate the causes.
  \item Come up with adaptive strategies to deal with the additional hyperparameters introduced by \cite{vieillard2020munchausen}.
\end{itemize}

\subsubsection{Unsupervised Skill Discovery / Curiosity}
Pretraining neural networks in an unsupervised setting showed to be extremely effective for language models. Similarly, RL agents can be pretrained in an environment with different goals in mind. Projects in this direction could (reimplement/) validate one of the papers below and extent their work with interesting ablation studies or algorithmic modifications.
\begin{itemize}
  \item \cite{Plan2Explore2020}
  \item \cite{DADS2020}
\end{itemize}

%\subsubsection{Policy Adaptation during Deployment}
%Training RL agents in simulation can accelerate learning substantially, and may sometimes be the only viable option due to hardware constraints. However, it is difficult to model the system in simulation accurately enough such that trained policies also work in the real world. \cite{Hansen2021} propose to close this sim2real gap using self-supervision during deployment.
%\begin{itemize}
%  \item Validate the proposed algorithm in a sim2sim setting.
%  \item If everything works correctly we can also run real world experiments with Justin's DLR Hand II.
%\end{itemize}

\subsubsection{NeRF based Kinematic Calibration}
Neuronal radiance fields are quite popular these days in the computer vision community. While they are often used for noval view synthesis, to create impossible camera effects with real world footage, they also provide a new way to calibrate camera extrinsics and intrinsics \cite{lin2021barf, Sucar:etal:ICCV2021, wang2021nerfmm, SCNeRF2021}.
This is possible because the extrinsics and intrinsics can be trained together with the model.

We still require tracking of visual markers to calibrate the kinematic of a robot (see the work of \cite{Birbach2014} as an example). 
However, implicit neuronal scene representations might enable us to learn the robot's kinematic end-to-end.

\begin{itemize}
  \item Try to utilize a sim-setup (pybullet) with a simple synthetic pinhole camera and a robot arm that holds the camera.
  \item The work of \cite{SCNeRF2021} should provide a starting point for a NeRF implementation in pytorch.
  \item The calibration can run offline, since NeRFs are still computational expensive.
  \item Try to incorporate the vector decomposition 'trick' of \cite{TensoRF} to reduce training/calibration time.
\end{itemize}



% \subsubsection{Compare Bullet/MuJoCo/Isaac by implementing DLR Hand II Environment}
% Need to have GPU.


\subsubsection{Procgen Benchmark}
Implement and experiment with different agents on this procedurally-generated benchmark \cite{procgen2020}. 
You can compare your results with other submissions on the leaderboard (\url{github.com/openai/procgen}).

\subsubsection{Comparing Different Methods to Handle POMDPs in the Form of Randomized Environments}
\textit{LSTM vs Stacking fuer Domain Noise}

\subsubsection{Casting Sim2Real as Meta-Reinforcement Learning}
The PEARL algorithm introduced by \cite{rakelly2019} promises sample-efficient Meta-Reinforcement Learning.
\textit{PEARL fuer Sim2Real aber dieses mal ohne Aufwand fuer die Env betreiben zu muessen und direkt from scratch implementiert.}

\bibliographystyle{apalike2}
\bibliography{minimal-research}

\end{document}
